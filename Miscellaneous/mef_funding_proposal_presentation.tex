\documentclass{beamer}
\usepackage[utf8]{inputenc}
\usetheme{Antibes}
\usecolortheme[RGB={223,36,153}]{structure}

\title{\emph{TMAMB} Reading Group \\ MEF Proposal Presentation}
\author{}
\date{July 4th, 2022}
\logo{\includegraphics[scale = 0.1]{mockingbird_logo.png}}

\begin{document}

\maketitle

\begin{frame}{Table of Contents}
\tableofcontents
\end{frame}

\section{About \emph{To Mock a Mockingbird}}
\begin{frame}{About \emph{To Mock a Mockingbird}}
  \begin{exampleblock}{}
  {\emph{To Mock a Mockingbird and Other Logic Puzzles: Including an Amazing Adventure in Combinatory Logic} is a book by the mathematician and logician Raymond Smullyan. It contains many nontrivial recreational puzzles of the sort for which Smullyan is well known. It is also a gentle and humorous introduction to combinatory logic and the associated metamathematics, built on an elaborate ornithological metaphor.}
\end{exampleblock}
\end{frame}

\section{About the Reading Group}
\begin{frame}{About the Reading Group}
  \begin{itemize}
  \pause
    \item The \emph{TMAMB} Reading Group exists to encourage and assist interested parties (``Junior Bird Mockers'') in working through the book. \pause
    \item The material is divided into weekly chunks. \pause
    \item Participants can discuss their solutions with, and solicit hints from, other members. \pause
    \item Office hours style sessions with members who have already mastered the material (``Certified Bird Mockers'') are available. \pause
    \item A new session begins with each UW academic term.
  \end{itemize}
\end{frame}

\begin{frame}{About the Reading Group (cont.)}
  \begin{itemize}
  \pause
    \item The group is organized on Discord, a voice chat and instant messaging platform. \pause
    \item Membership is open to anyone. \pause
    \item There are currently about 30 members, including UW Math undergrads, grad students, faculty, and alumni; other members of the UW community; and members not affiliated with UW.
  \end{itemize}
\end{frame}

\section{Benefits to Math Undergrads}
\begin{frame}{Benefits to Math Undergrads}
  \begin{itemize}
  \pause
    \item A challenging but accessible head start to the material in CS 245/360/442, PMATH 330/432/433, and PHIL 240/257. \pause
    \item A positive social environment with networking opportunities. 
  \end{itemize}
\end{frame}

\section{Proposed Spending}
\begin{frame}{Proposed Spending}
  \begin{itemize}
  \pause
    \item To date, the group has been a volunteer effort with all expenses paid personally by its organizers. \pause
    \item The proposed funding will enable the \emph{TMAMB} Reading Group to provide participants in the September session with a free personal copy of \emph{TMAMB}. \pause
    \item No other significant expenses are anticipated. \pause
    \item Any funding in excess of the proposal could be used to \begin{itemize} \item advertise the next session, \item organize a public lecture on combinatory logic at UW, \item or expand the scope of the group to cover additional material. \end{itemize}
  \end{itemize}
\end{frame}

\begin{frame}{}
  \begin{center}
  \Large Thank you!
  \end{center}
\end{frame}

\end{document}
