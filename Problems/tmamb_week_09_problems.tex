\documentclass[12pt, letterpaper]{article}
\usepackage{problems}
\title{\emph{To Mock A Mockingbird} Reading Group\\Week 9 Problems}
\begin{document}
\maketitle

\begin{prob}{14.1} 
Prove that if 1) if $y$ sings on a given day, then $Pxy$ sings too; 2) if $x$ doesn't sing on a given day, then $Pxy$ sings too; 3) if $x$ and $Pxy$ both sing on a given day, then $y$ sings too; and 4) for every bird $x$ there is a bird $y$ such that $y$ sings on those and only those days on which $Pyx$ sings; then all birds sing on all days. (Chief Bird Mocker's Note: This can actually be shown without using the first assumption.)
\end{prob}

\begin{prob}{14.2} 
Does the same conclusion follow from 1), 2), and 3) plus the presence of a lark? How about a cardinal? Both a lark and a cardinal?
\end{prob}

\begin{prob}{14.3} 
Can you find a single combinatorial bird whose presence, plus 1), 2), and 3), would imply that all birds sing on all days?
\end{prob}

\begin{prob}{14.E1}
Assuming 1), 2), and 3), prove that for all birds $x$, $y$ and $z$, 
\end{prob}

\begin{tabular}{r l}
    \textbf{a)} & $Pxx$ sings on all days. \\
    \textbf{b)} & If $Py(Pyx)$ sings on all days, so does $Pyx$. \\
    \textbf{c)} & If $Pxy$ and $Pyz$ sing on all days, so does $Pxz$. \\
    \textbf{e)} & If $Px(Pyz)$ sings on all days, so does $Py(Pxz)$.
\end{tabular}

\vspace{6pt}
\noindent (Chief Bird Mocker's Note: The book's version of this exercise also has a part d), but it asks you to prove something untrue! See if you can find the problem with it.)

\begin{prob}{14.E2}
Show that if there is a bird $P$ such that a) for any birds $x$ and $y$, if $Px(Pxy)$ is lively, so is $Pxy$; b) for any birds $x$ and $y$, if $x$ and $Pxy$ are both lively, so is $y$; and c) for any bird $x$ there is a bird $y$ such that $Py(Pyx)$ and $P(Pyx)y$ are both lively; then all birds are lively.
\end{prob}

\begin{prob}
Suppose that a bird's being lively means it sings on all days. Show that 1), 2), 3) and 4) from 14.1 imply a), b), and c) from the previous exercise.
\end{prob}

\begin{prob}{15.1} 
Suppose there is a bird $a$ such that for any bird $x$, the bird $ax$ sings on those and only those days on which $xx$ sings, and also that for any bird $x$, there is a bird $x'$ such that for every bird $y$, the bird $x'y$ sings on those and only those days on which $xy$ does not sing. Why are these two assertions logically incompatible?
\end{prob}

\begin{prob}{15.2} 
Suppose instead there is a bird $N$ such that for any bird $x$, $Nx$ sings on those and only those days on which x does not sing, and that the forest also contains a sage bird. Why are these two assertions also logically incompatible?
\end{prob}

\begin{prob}{15.3} 
Now suppose instead that there is a sage bird and a bird $A$ such that for any birds $x$ and $y$, $Axy$ sings on those and only those days on which neither $x$ nor $y$ sings. Are these two assertions logically incompatible?
\end{prob}

\begin{prob}{16.1} 
Prove that if 1) for any birds $x$ and $y$, if $exy$ sings on a given day, so does $y$; 2) for any birds $x$ and $y$, $x$ and $exy$ never sing on the same day; 3) for any birds $x$ and $y$, $exy$ sings on all days on which $x$ doesn't sing and $y$ does sing; and 4) for any bird $x$ there is a bird $y$ such that $y$ sings on the same days as $eyx$; then none of the birds of this forest ever sing.
\end{prob}

\begin{prob}{17.1} 
Prove that if 1) all nightingales (of this forest) sing; 2) $x'y$ sings if and only if $xy$ doesn't sing; 3) $x*y$ sings if and only if $x(yy)$ sings; and 4) $\mathcal{N}x$ sings if and only if $x$ is a nightingale; then there is a bird $\mathcal{G}$ in this forest which sings, but which is not a nightingale. Name it explicitly in terms of $\mathcal{N}$, $'$, and $*$. (Chief Bird Mocker's Note: It is not given that $x'' = x$ but you may assume it if you like.)
\end{prob}

\begin{prob}{17.2} 
Name a bird $\mathcal{G}_1$ which sings but is not a nightingale, but which might or might not be the same bird as $\mathcal{G}$.
\end{prob}

\begin{prob}{17.3} 
A bird $A$ \emph{represents} a set $\mathcal{S}$ of birds when $Ax$ sings if and only if $x$ is a member of the set $\mathcal{S}$. A set of birds is a \emph{society} when it is represented by some bird. Determine whether the set of singing birds constitutes a society using 2) and 3) from 17.1. Also, use 3) to prove that every society must either contain at least one bird that sings or lack at least one bird that doesn't, and explain what this has to do with whether the singing birds constitute a society.
\end{prob}

\end{document}