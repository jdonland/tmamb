\documentclass[12pt, letterpaper]{article}
\usepackage{problems}
\title{\emph{To Mock A Mockingbird} Reading Group\\Week 7 Problems}
\begin{document}
\maketitle

\disclaimer

\begin{prob}{12.1}   
Derive a double mockingbird $M_2xy = xy(xy)$ from $B$ and $M$.
\end{prob}

\begin{prob}{12.2}   
Derive $L$ from $B$, $C$, and $M$ or $B$, $R$, and $M$.
\end{prob}

\begin{prob}{12.3}   
Derive $L$ from $B$ and $W$.
\end{prob}

\begin{prob}{12.4}   
Derive $L$ from $M$ and $Q$.
\end{prob}

\begin{prob}{12.5}   
Derive a converse warbler $W'xy = yxx$ from $B$, $T$, and $M$. (Hint: Use $B$, $R$, and $M$.)
\end{prob}

\begin{prob}{12.6}   
Derive $W$ from $B$, $R$, $C$, and $M$ using only four letters.
\end{prob}

\begin{prob}{12.7}   
Express $W$ in terms of $B$, $T$, and $M$, using ten letters. (Note: There are two ways.)
\end{prob}

\begin{prob}{12.8}   
Is $M$ derivable from $B$, $T$, and $W$?
\end{prob}

\begin{prob}{12.9}   
Are $W^*xyz = xyzz$ and $W^{**}xyzw = xyzww$ derivable from $B$, $T$, and $M$?
\end{prob}

\begin{prob}{12.10}  
Derive a hummingbird $Hxyz = xyzy$ from $B$, $T$, and $M$. (Hint: Use $B$, $C$, and $W$.)
\end{prob}

\begin{prob}{12.11}  
Derive $W$ from $R$ and $H$.
\end{prob}

\begin{prob}{12.12}  
Use $G$ to express $Sxyz = xz(yz)$ in terms of $B$, $C$, and $W$ using six letters.
\end{prob}

\begin{prob}{12.13}  
Is $H$ derivable from $S$ and $R$?
\end{prob}

\begin{prob}{12.14} 
Express $W$ in terms of $S$ and $R$ and in terms of $S$ and $C$.
\end{prob}

\begin{prob}{12.15}  
Express $W$ in terms of $T$ and $S$.
\end{prob}

\begin{prob}{12.16}  
Derive $M$ from $T$ and $S$.
\end{prob}

\begin{prob}{12.E1} 
\end{prob}

\begin{tabular}{r l}
\textbf{a)} & Derive $G_1xyzwv = xyv(zw)$ from $B$ and $T$. \\
\textbf{b)} & Derive $G_2xyzw = xw(xw)(yz)$ from $G_1$ and $M$. \\
\textbf{c)} & Derive $I_2x = xII$ from $B$, $T$, and $I$. \\
\textbf{d)} & Prove $I_2(Fx) = x$. \\
\textbf{e)} & Prove that $G_2F(QI_2)$ is a warbler.
\end{tabular}

\begin{prob}{12.E2}  
Prove that $B(B(BW)C)(BB)$ is a starling.
\end{prob}

\begin{prob}{12.E3}  
Derive a phoenix $\Phi xyzw = x(yw)(zw)$ from $S$ and $B$ using four letters.
\end{prob}

\begin{prob}{12.E4}  
Derive a psi bird $\Psi xyzw = x(yz)(yw)$ from $B$, $C$, and $W$. (Hint: Use $H^* = BH$ and $D_2$.)
\end{prob}

\begin{prob}{12.E5} 
\end{prob}
\begin{tabular}{r l}
\textbf{a)} & Derive $\Gamma xyzwv = y(zw)(xyzwv)$ from $\Phi$ and $B$. \\
\textbf{b)} & Derive $\Psi$ from $\Gamma$ and $K$.
\end{tabular}

\vspace{6pt}
\noindent (Chief Bird Mocker's Note: Beware! There's a typo in the formula for $\Gamma$ in the book. It's been corrected here.)

\begin{prob}{12.E6} 
\end{prob}
\begin{tabular}{r p{4.5in}}
\textbf{a)} & Derive $S'xyz = yz(xz)$ from $S$ and a bird already derived from $B$ and $T$. \\
\textbf{b)} & Derive $W$ from $S'$ and $I$.
\end{tabular}

\begin{prob}{12.E7}  
Derive a bird $\hat{Q}$ from $Q$ alone such that $C\hat{Q}W$ is a starling using six letters.
\end{prob}
\end{document}