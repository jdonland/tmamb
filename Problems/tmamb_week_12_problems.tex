\documentclass[12pt, letterpaper]{article}
\usepackage{problems}
\title{\emph{To Mock A Mockingbird} Reading Group\\Week 12 Problems}
\begin{document}
\maketitle

\disclaimer

\begin{prob}{24.1}
Let $n^+$ denote the successor of a natural number $n$ and let $\noverline{0} = I$ and $\noverline{n^+} = Vf\noverline{n}$. Prove that if $n \neq m$ then $\noverline{n} \neq \noverline{m}$. 
\end{prob}

\begin{prob}{24.2}
Find a bird $P$ such that $P\noverline{n^+} = \noverline{n}$.
\end{prob}

\begin{prob}{24.3}
Find a bird $Z$ such that $Z\noverline{0} = t$ and $Z\noverline{n} = f$ if $n \neq 0$. 
\end{prob}

\begin{prob}{24.4}
Find a bird $A$ such that for any birds $x$ and $y$, $A\noverline{0}xy = x$ and $A\noverline{n}xy = y$ if $n > 0$. 
\end{prob}

\begin{prob}{24.5-7}
Explain how to find birds $\oplus$, $\otimes$, and $\oexp$ such that $\oplus\noverline{n}\noverline{m} = \noverline{n + m}$, $\otimes\noverline{n}\noverline{m} = \noverline{n \times m}$, and $\oexp\noverline{n}\noverline{m} = \noverline{n^m}$.
\end{prob}

\begin{prob}{24.8}
A bird $A$ is a \emph{property} bird when $A\noverline{n}$ is a propositional bird. A bird property bird $A$ \emph{computes} a set of numbers when $A\noverline{n} = t$ if $n$ is a member of the set and $A\noverline{n} = f$ otherwise. A set of numbers is \emph{computable} when there is a bird that computes it. Explain how to find a bird $A$ that computes the set of even numbers.
\end{prob}

\begin{prob}{24.9}
A bird $A$ is \emph{relational} when $A\noverline{a}\noverline{b}$ is a propositional bird. Explain how to find a bird $g$ such that $g\noverline{a}\noverline{b} = t$ if $a > b$ and $g\noverline{a}\noverline{b} = f$ otherwise. (Hint: $>$ is the only relation such that 1) if $a = 0$ then $a > b$ is false, 2) if $a \neq 0$ and $b = 0$ then $a > b$ is true, and 3) if $a \neq 0$ and $b \neq 0$ then $a > b$ is true if and only if $a - 1 > b - 1$.)
\end{prob}

\begin{prob}{24.10}
A relational bird $A$ is \emph{regular} if for every number $n$, there is at least one number $m$ such that $A\noverline{n}\noverline{m} = t$. A bird $A'$ is a \emph{minimizer} of a regular relational bird $A$ if for every number $n$, $A'\noverline{n} = \noverline{k}$ where $k$ is the smallest number such that $A\noverline{n}\noverline{k} = t$. Prove that every regular relational bird has a minimizer. (Hint: First explain how to find a bird $A_1$ such that 1) if $A\noverline{n}\noverline{m} = f$ then $A_1\noverline{n}\noverline{m} = A_1\noverline{n}\noverline{m^+}$ and 2) if $A\noverline{n}\noverline{m} = t$ then $A_1\noverline{n}\noverline{m} = \noverline{m}$. Then let $A' = CA_1\noverline{0}$ and show that it minimizes $A$.)
\end{prob}

\begin{prob}{24.11}
The \emph{length} of a number its number of digits. Find a bird $\ell$ such that if $n$ has length $k$ then $\ell\noverline{n} = \noverline{k}$. (Chief Bird Mocker's Note: Be careful about the length of 0! The provided solution to this problem overlooks this subtlety.)
\end{prob}

\begin{prob}{24.12}
The \emph{concatenation} of $n$ and $m$, denoted $n \ast m$, is the number obtained by writing the digits of $n$ followed by the digits of $m$. Explain how to find a bird $\ocoasterisk$ such that $\ocoasterisk\noverline{a}\noverline{b} = \noverline{a \ast b}$.
\end{prob}

\noindent(Chief Bird Mocker's Note: The third rule for logical equality in the preamble of Chapter 25 contains an error. It should read: ``If we can prove $X = Y$, then for any term $Z$ we can conclude that $XZ = YZ$ and $ZX = ZY$.")

\begin{prob}{25.1}
A \emph{term} is $S$, $K$, or $(XY)$ where $X$ and $Y$ are each terms. Every term names a bird. A \emph{sentence} is $X = Y$ where $X$ and $Y$ are each terms. $X = Y$ is true if and only if $X$ and $Y$ are two names for the same bird. The \emph{G\"odel numbers} of the symbols $S$, $K$, $($, $)$, and $=$ are 1, 2, 3, 4, and 5 respectively. The G\"odel number of a term or a sentence is the concatenation of the G\"odel numbers of its symbols. A \emph{numeral} is any of the birds $\noverline{0}$, $\noverline{1}$, $\noverline{2}$... etc. Let $n^\#$ denote the G\"odel number of the numeral $\noverline{n}$. Let $s$ denote the G\"odel number of the bird $\sigma$ such that $\sigma \noverline{n} = \noverline{n^+}$. Then $(n^+)^\# = 3 \ast s \ast n^\# \ast 4$. Explain how to find a bird $\delta$ such that $\delta \noverline{n} = \noverline{n^\#}$.
\end{prob}

\begin{prob}{25.2}
The \emph{G\"odel numeral} of an expression $X$, denoted $\ulcorner X \urcorner$, is the numeral representing the G\"odel number of $X$. The \emph{norm} of $X$ is $X\ulcorner X \urcorner$. Derive a normalizer bird $\Delta \noverline{n} = \noverline{n \ast n^\#}$ using $\ocoasterisk$ and $\delta$. (Chief Bird Mocker's Note: The provided solution to this problem is needlessly complicated. See if you can find a more elegant one!)
\end{prob}

\begin{prob}{25.3}
A term $X$ \emph{designates} a number $n$ if the sentence $X = \noverline{n}$ is true. A term is \emph{numerical} if it designates some number. Prove that for any term $A$, there is a term $X$ such that $A\ulcorner X \urcorner = X$ is true and explain why this implies that some numerical term must designate its own G\"odel number. (Chief Bird Mocker's Note: This one is tough! Try first finding a term $Y$ such that $Y\ulcorner Y \urcorner = A(\Delta \ulcorner Y \urcorner)$.)
\end{prob}

\begin{prob}{25.4}
A sentence $X = Y$ is a \emph{G\"odel sentence} of a set of numbers $\mathscr{S}$ when it is true if and only if its G\"odel number is in $\mathscr{S}$. Prove that every computable set has a G\"odel sentence. (Hint: If $n$ is the G\"odel number of $X$, then $n \ast 52$ is the G\"odel number of $X = t$. First prove that if $\mathscr{S}$ is computable, then the set $\mathscr{S}^*$ consisting of all $n$ such that $n \ast 52$ is in $\mathscr{S}$ is also computable.)
\end{prob}

\begin{prob}{25.5}
Prove that if $\mathscr{S}$ is computable, then the set $\mathscr{S}'$ consisting of all numbers not in $\mathscr{S}$ is also computable.
\end{prob}

\begin{prob}{25.6}
Answer the Grand Question: Prove that the set $\mathscr{T}$ consisting of the  G\"odel numbers of all true sentences is not computable.
\end{prob}

\end{document}
